
\section{Instrument Data Acquisition and Control System}


\subsection{Introduction}

The SNS Instrument Data Acquisition and Controls System serves as
the interface between the beam line instrument equipment and the
experimentalist User and the data analysis systems. The System is
responsible for acquiring data from neutron detectors (including
linear position sensitive $^3$He tubes, crossed-fiber detectors,
and Anger cameras) and  beam monitors, and experiment data including
neutron chopper phase, environmental conditions of the sample,
and other relevant parameters as required for the neutron sciences
experimental program. The data acquired must be saved to a central
repository using the NeXus~\footnote{``NeXus Scientific Data Format,''
\texttt{http://www.nexusformat.org}} data format, cataloged and
archived in a way suitable for use by data analysis programs.  The System
must also provide control for the experiment setup. Points of control
include chopper setting for energy selection, motors for positioning and
orienting the sample or the detector assembly, and a variety of sample
environment equipment including close cycle refrigerators, furnaces,
load frames, magnetic or electric field generators, and other equipment
as required by the experiment. A block diagram of the system in shown
in Fig.~\ref{blockdiagram}. Additionally, the System must interface
with information systems residing in relational databases to acquire
information from the proposal tracking system (IPTS), sample information
(ITEMS) and other systems.

\begin{figure}[htb]
	\centering
%	\includegraphics*[width=4in]{concept.eps}
	\caption{Simplified block diagram showing the key components to
	the Instrument Data Acquisition and Control System and primary
	data flow.} 
	\label{blockdiagram}
\end{figure}

\subsubsection{Scope}
The scope of this section includes:

\begin{itemize}

\item the computer interface to the detector electronics

\item the interface between the accelerator timing system and the
detector electronics

\item the interface to the sample environment equipment and other controls
points for the instrument, i.e. ``slow controls''

\item the mechanism for automating experiment control, i.e. ``scans''

\item the experiment graphical user interface 

\item and the tools to generate the NeXus data files and to archive and
catalog these data files.

\end{itemize}

The section of the upgrade plan includes software required for these
systems, and the computer systems necessary to run these systems. Also
included is the network architecture for the instruments and the data
systems which is necessary to implement this plan.

The electronics which interface to the detector assemblies are not
within the scope of this upgrade plan. However, this plan should factor
in the critical need for a future upgrade in this area, and provide an
appropriate upgrade path.

Data analysis software is covered eslewhere in this upgrade plan.
However, this section should be informed by design of the current data
analysis system and for concurrent upgrade plans in this area. The ability
to display live data overlaps with the data acquisition and data analysis
areas so is partially included here. The ability to support feedback from
the data analysis and data reduction systems is not directly within the
scope of this project, however this should be considered a long range goal
with this current upgrade providing the infrastructure to enable this.
The data acquisition and data analysis systems should be well integrated
from the Users perspective.

\subsubsection{Design Principles}

The design for the upgrade to the System needs to consider that the SNS
operational period is measured in decades. Technologies should be chosen
such that there is a reasonable expectation of support for greater then
five years, and preferably well beyond that.  Use of proprietary software
or hardware which may have a limited support lifetime should avoided where
possible. The design should support incremental improvements and upgrades
over time to be able to adapt to emerging technologies where appropriate.
However, the design should make use of proven technologies and techniques
to ensure a highly reliable and maintainable System.

Software and system designs which are in use at similar facilities
should be used where appropriate.  Existing software from other
facilities meeting our requirements should be used or built upon where
appropriate.  Reimplementation of existing software should only be done
with a compelling reason.  Collaboration with other neutron sources,
synchrotron sources, or other experimental facilities is encouraged.
An overview of data acquisition and instrument control systems at neutron
and synchrotron sources is included in the appendix. Links are included
for additional information.

A ``toolkit'' based approach is desired. General use (generic) tools
should be identified or developed to meet general needs across the
instrument suite. Customizations specific to individual instrument should
be configurations using general purpose tools or optional add-ons.
Additionally, newly developed or expanded tools should be designed
(where reasonable) such that they could be used outside of SNS. In the
same manner that this upgrade will take advantage of software and tools
developed at other facilities, developers here should be aware that
tools being developed may provide value to other facilities.

All software must be stored in a shared repository, with version
control and clearly defined release tags for production versions.
Automatic testing should be used where reasonable.

Documentation must be provided to support the ongoing operation and
support of the System.

\subsection{Detector Interface}
\subsubsection{Requirements}

The computer interface to the detectors must be capable of supporting
data rates of 10 million events per second (at least equivalent to existing
design).
\todo{Event Rate}{%
10 Msps sufficient?}

The interface must provide a network stream of data to one 
\todo{Network Data Stream}{One always sufficient?}

listeners at this data rate. The stream must provide accelerator
pulse information (charge) and neutron event information (pixel ID
and time of flight).

The interface must be able to provide the ability to configure detector
electronics (e.g. high voltage levels, thresholds, discriminator levels,
etc.). Preferably, this should use the same user interface tools as the
experimental control system.

The interface should provide status information regarding the detector
assemblies and make this data available in a way that can be displayed
on the experiment control user interface or other monitoring systems.
Sufficient status information should be available so as to facilitate
in situ detector troubleshooting.

It is preferable that data processing occur in the hardware (FPGA) level
to ensure consistent and predictable response with well defined processing
rates. However, the system should be capable of handling the input of raw
data with data processing within the computer system. This may be required
for detector development and upgrades and can provide for prototyping
solutions as has been used recently with the Anger camera detectors.

\todo{Histogram Mode Necessary?}{%
The interface must be capable of supporting either event mode or
histogramming mode, with the ability to switch between these modes on
an experiment by experiment level}

Additional information concerning the detector read-out boards (ROCS) and
the interface to those electronics can be found in Interface Standards
document prepared by Lloyd Clonts (most recent version December 13,
2011 circulated as Draft).


\subsubsection{Implementation Plan}

\begin{enumerate}

\item Refine existing streaming network protocol such that in can support
slow controls data in addition to neutron event data.

\item Investigate commercial solutions for computer interface to detector
electronics. Possibility includes an evaluation board which could replace
the `DSP' module and a GigE network card to replace the `OCC' card.
TBD: hardware design.

\item TBD: software.

\item TBD: PVs for integration with UI and scanning service

\end{enumerate}

\subsection{Timing Interface}
\subsubsection{Requirements}

The Timing subsystem receives data from the accelerator timing system
over the dedicated Real Time Data Link (RTDL) and Event Link (EL). These
data include the 60~Hz (nominal) synchronization clock, pulse ID (time
stamp), proton charge per pulse and other accelerator information.


\begin{itemize}
\item Decode data link and provide a synchronized clock to the detector
electronics for time stamping neutron events with time of flight.  \item
Provide synchronized triggers in reference to proton beam on target to
choppers and other synchronous systems \item Provide proton charge per
pulse and other accelerator parameters to the detector computer interface
for inclusion in the neutron event data stream.

Additional information in document ``Neutron Event Timing'' prepared by
Steve Hicks.

\end{itemize}

\subsubsection{Implementation Plan}

\begin{enumerate}

\item Incorporate existing FPGA code from accelerator control system
timing receiver card into `DSP' interface board of data acquisition
system.

\item In conjunction with detector interface upgrade (above), develop
simplified, consistent hardware interface module

\item TBD: Define software requirements.

Detailed plan is being prepared by team of Steve Hicks, Lloyd Clonts,
Jeff Patterson and others.

\end{enumerate}

\subsection{Slow Controls and Sample Environment Equipment Interface}
\subsubsection{Requirements}

Sample environment equipment consists of a large number of commercial
and custom equipment. Much of the equipment provides a network interface
for remote monitoring and control. Some of the equipment provides an
interface of another bus, such as GPIB or serial communication.

Motor controllers are primarily of type Parker 6K from Compumotor,
although there are a few exceptions.

Some sample environment equipment is dedicated to a particular beam line.
Other equipment may be shared between a few or all beam lines.

The System must be capable of supporting changing of sample environment
equipment on a routine basis.

\subsubsection{Implementation Plan}

\begin{enumerate}
\item Setup a development system in the H-104 lab. Install EPICS
base (latest production version) and needed support modules
and extensions. Begin with simple hardware setup, such as a
Parker 6K motor controller (EPICS device support exists) and a
Lakeshore temperature controller (EPICS device support exists for
many models).  Create demonstration softIOC using standard EPICS
toolkit\footnote{``Experimental Physics and Industrial Control System
toolkit,'' \texttt{http://www.aps.anl.gov/epics} }.

\item Begin to develop framework for support of dynamic sample environment
equipment setups.

\item Identify existing sample environment equipment with existing EPICS
support. Demonstrate integration on test stand.

\item Identify existing equipment without existing EPICS support.
In each case, identify appropriate path towards integration, for example,
develop new device driver or use LabView to EPICS interface.

\item Determine mechanism for Channel Access client to provide data from
slow control system to data translation process and to non-EPICS live
viewing tools (see ``Stream Management Service'' document prepared by
David Dillow).

\end{enumerate}

\todo{Channel Finder}{%
Can BNL NSLS-2's ``Channel Finder'' provide a useful mechanism for
flagging which PVs are available at a given beam line? Scan service and
UI could potentially use this to discover which parameters are available.
Otherwise, XML text files?
}

\todo{Non-Standard Equipment}{%
integration for non-standard equipment
}

Note: some specialized sample environment equipment does not provide a
remote interface appropriate for integration into a standard controls
system (for example, the rheometer for Liquid Reflectometer or the
MTS load frame for VULCAN). Such devices will continue to present a
difficulty regardless of the control system design. Going forward,
it is important for such equipment, whether purchased by a particular
instrument or meant for general use, is reviewed for its ability to be
integrated into a standard control system.


\subsection{Automated Experiment Control}
\subsubsection{Requirements}

A mechanism for automating experiment control is required. Such a
mechanism is commonly called a ``scripting engine'' or a ``scanning
service.''

The scanning service must be able to accept batch jobs submitted through
a network interface. It must be able to queue multiple jobs and run them
sequentially, with the ability to add or remove jobs from the queue,
and to modify jobs which have been submitted but not yet run. It must
be able to pause or abort a job. It must provide a mechanism to monitor
the status of the queue through the user interface application.

It should be possible for multiple user interface applications to interact
with the scanning service.

There should be multiple ways to create a scan. Creating a scan must
not require the use a general purpose programming language, although
such functionality could be provided for optional use. A domain specific
language or a library providing a sufficiently simplified interface
is preferred.  An optional
graphical user interface for creating scan is preferred. A mechanism
such as a table or spreadsheet view for creating scans is required.

To facilitate use by inexperienced external users, the system should
have reasonable defaults (such as automatically save data at run end,
and reasonable responses to normally anticipated fault conditions).

\subsubsection{Implementation Plan}

A prototype of a scan engine has been developed in Java using the
eclipse framework and CSS. Unit tests for current functionality exists.
The prototype network interface for submitting scans uses RMI.

\begin{enumerate}

\item Begin integration with the EPICS lab test stand to demonstrate
functionality.

\item Explore options for collaboration with other Laboratories, such
as APS or NSLS-2.

\item Gather user input for additional functional requirements.

\item Identify potential replacements for RMI to allow non-Java
applications to interface with the scanning server, such as for Mantid
or Python applications to submit scan jobs.

\item Integrate with data stream ``Aggregator'' (see ``Stream Management
Service'' document prepared by David Dillow) to provide slow controls
data to Translation Service. 

\end{enumerate}

\todo{Run Control}{%

TBD: define reasonable system defaults and error responses

TBD: overall experiment control (proposal number, run number, start/stop, etc.)

TBD: PVs for communication with detector systems, choppers, etc

TBD: PVs for communication of status information with Aggregator 
and translation service

TBD: integration with IPTS and other static data sources
}

\subsection{User Interface}
\subsubsection{Requirements}

A common, integrated user interface for experiment control is preferred.
The use of multiple independent applications with their own GUI should
be avoided.

The user interface must be independent from experiment control. Closing
a user interface application should not cause an experiment to end. It
should be possible for an experiment to run with zero, one or multiple
user interface programs running for the experiment.

A "shell" interface is valued by some Users, but Users should not be
required to use it.

There should be an abstraction layer for experiment perspective
(i.e. chopper phase setting in wavelength, motors for crystal movement
in $(h,k,l)$ space, etc.) The user interface should support multiple ways
of presenting this data to the user based on experiment needs. This
abstraction does not necessarily need to be implemented in the user
interface, but the implementation should be such that it is adaptable
to user needs.

\subsubsection{Remote Monitoring}

It is desirable to provide a read-only remote monitoring capability for
experiment status through a commonly available interface such as a web
browser. Such a remote status overview could provide information as to
the current conditions of the sample environment, current run number,
status of scan queue, data rates and other generic information to inform
the User as to the status of the experiment.

\todo{Remote Monitoring}{%
network infrastructure requirements and networking policy implications
}

\subsubsection{Fault Notification}

The integrated user interface should provide a clear display in experiment
hutch as to availability of equipment necessary to conduct experiment.

It is also desirable to provide a summary display for all instruments
available to IHC for off-hours monitoring.

A mechanism for alarming or remote notification is desirable. A consistent
mechanism should be used for providing fault notification throughout
the system, i.e. detectors, choppers, sample environment equipment, etc.

\subsubsection{Implementation Plan}

\begin{enumerate}

\item As the EPICS slow controls prototype is developed,
create displays using existing Control System Studio
(CSS) toolkit~\footnote{``Control System Studio toolkit,''
\texttt{http://cs-studio.sourceforge.net/docbook}}. Develop standard
nomenclature, color rules, font rules, and directory structure.


\item Demonstrate suitability of CSS WebOPI tool for providing remote
monitoring capability.

\item Investigate suitability of CSS BEAST tool for alarming and fault
notification.

\end{enumerate}
\todo{Scientific Control}{%
TBD: abstraction layer to meaningful ``physics'' units

TBD: archiving of status information, i.e. non experiment data or
engineering information

TBD: alarm configuration for fault notification and integration of 
various subsystem into EPICS to be able to provide this service.
}

Additional description of requirements and implementation plan details
can be found in the document "SNS Data Acquisition and Control GUI
Requirements Analysis" (Chen and Kasemir).

\subsection{Interface to Neutron Data and Live Data Visualization}

\subsubsection{Requirements}

Rich live viewing of experiments will be provided in
Mantid~\footnote{``Mantid,'' \texttt{http://www.mantidproject.org}}.
This will leverage the existing capabilities within Mantid for visualizing
data. It will provide the capability to perform data normalization,
background subtraction and other manipulation to present data during the
experiment in a meaningful way. Details of live viewing within Mantid
are covered elsewhere in this document.

However, the data acquisition system will also need to provide a basic
live data visualization capability and interaction with the neutron data
stream. This is required to facilitate sample alignment or orientation,
and to provide a mechanism for automated scanning based on, for example,
number of counts within a region of interest.

The System must provided variables accessible to the user interface and
to the scanning service to provide number of events (i.e. counts on the
detector system). This variable should be resettable to zero. A variable
representing count rate is desired.

A live 2D plot representing a simplified view of the physical detector
assembly is required.  Such a plot shows X and Y pixel position,
integrated over time of flight with color representing intensity
(i.e. number of counts).  A live 2D plot representing X and time of
flight, integrated over Y is also required.  It is preferred that the
color scale representing intensity be configurable.  A profile of the
plot should be provided. The ability to select a sub-region for viewing
the profile is desired.

Variations on these plots and other basic plots plots may also be
desired. For example, representing the X dimension as the~2$\theta$
angle or providing a plot of S(Q,$\omega$) or D-space. However, Mantid
should be used where reasonable rather then re-implementing existing
features already available within Mantid.

The ability to select a region of interest from the 2D plot is desired.
The region can be a rectangular area. It is desirable to be able to
acquire statistics from within this region of interest, such as the number
of detector counts, for use as a parameter within the scanning service.

\subsubsection{Implementation Plan}

\begin{enumerate}

\item Basic viewing functionality should be available within the user
interface used for experiment control.

\item More advanced viewing and analysis should be implemented in Mantid.
Development of ``live viewing'' capability should be done in parallel
with other development. It may also be worth pursuing using partial
NeXus files (i.e. mini-NeXus) as a data source for Mantid.

\item A listening service will need to be developed to listen to the
neutron data stream and histogram data to provide basic plotting. The
histograms will be converted into EPICS arrays for network transport
to the user interface program. CSS/BOY currently has the capability of
plotting such waveforms and calculating a profile. The 2D array can
be packaged as is typical within EPICS for 2D plots (such as image
waveforms).

\item Additional development is required to define and interact with
a region of interest or to select a slice of the waveform for viewing
of profile.

\item Additional user input is required to refine requirements.

\end{enumerate}
\todo{Histogramming Service}{%
TBD: refine requirements for histogramming service and it's interaction
with the data stream Aggregator service.

TBD: dividing line between this live view and that implemented for Mantid.
}
\subsection{Nexus file generation}
\subsubsection{Requirements}

Data should be available in repository for analysis withing seconds to
minutes from end of experiment. Ideally, partial data available will be
available during the run. Data is archived with a catalog in a relational
database for later retrieval.

\subsubsection{Archiving and Catalog}

The data should be archived with a database catalog to link various data
related to the experiment and provide a means for analysis and download
tools to locate and access the stored data.
\todo{ICAT}{%
TBD: ICAT3~\footnote{``ICAT Project,''
\texttt{http://www.icatproject.org/}}. 

TBD: Data download
}

\subsubsection{Implementation Plan}

\begin{enumerate}

\item Streaming translation service will listen to aggregated data
stream providing all neutron event data, slow control data and other
experimental parameters which need to be included in output files. See
``Stream Management Service'' document prepared by David Dillow.

\item Finalize network protocol for streaming data. 

Detailed plan for streaming NeXus generation
can be found in the document "Streaming Translation Service" by Kohl, et al.

\end{enumerate}

\todo{ICAT}{%
TBD: ICAT3 integration

TBD: translation monitoring

TBD: data download options (topcat?)
}

\subsection{Network Architecture}

\subsubsection{Requirements}

\todo{Network Architecture Requirements}{%
TBD
}

data flow, high throughput needs

segmentation between beam lines. No beam lines should be able to impact
other beam lines.

segmentation between external facing resources and internal resources.

Security: assume instrument controls are not secure so must be protected

Remote access for SNS staff to monitor, troubleshoot and repair.

Read-only remote status information for Users.

Reliability: All necessary resources should be located within the data systems
network. No unnecessary external dependencies.

\subsubsection{Implementation Plan}

\todo{Fiber Installation}{%
Installation of fibers connecting 8700 A103 to
8600 J115 

fibers from instruments to A103

General network logical architecture and topology
}

\subsection{Information Systems}

\subsubsection{IPTS}

\subsubsection{ITEMS}

\subsubsection{Experiment Scheduling System}

\subsubsection{Implementation Plan}

Dedicated Oracle RDB to be set up during maintenance period in January
2012. This database will host the current SCPROD applications which
includes ICAT. This Oracle instance will also be used to replicate IPTS
and ITEMS from XPROD which is on the ORNL Admin Protection Zone.  This is
intended to provide system resilency be removing external dependencies
for the data acqisition and translation/cataloging systems.


